\documentclass[11px]{specification}

% Input encoding and typographical rules for English language
\usepackage[utf8]{inputenc}
\usepackage[english]{babel}
\usepackage[english]{isodate}

\usepackage{float}

\usepackage{lipsum}

\usepackage{fontspec}
\usepackage{newunicodechar}
\newfontfamily\symbola{Symbola}
\newcommand{\downzigzagarrow}{\begingroup\symbola ↯\endgroup}
\newunicodechar{↯}{\downzigzagarrow}


% These define global texts that are used in headers and titles.
\title{BBB, Bestest Button Brick \\- Design Description}
\author{Jonatan Gezelius}
\date{Februari 2026}
\revision{Revision 0.1}
\companylogo{↯ Jegatron Electronics ↯}



\begin{document}

\maketitle

\section{Background}


\begin{multicols}{2}

\begin{colblock}
    \includegraphics[width=0.8\linewidth]{BBB-photo}
    \captionof{figure}{First iteration of BBB}
    \label{fig:systemDiagram}
\end{colblock}

The BBB is a multipurpose user programmable input device and also serves as an exercise product to hone my skills and try out different circuit solutions. Not all features and component make sense from a real product perspective, but serves as a learning experience.



It all started with the desire to have a more professional feel to my hobby and learning projects. So instead of using the classical tactile buttons forced down into a breadboard entangled in jumper wires, I bought some random mechanical keyboard switches and 3d-printed an enclosure for them. Then came the time to make the circuit board to empower this.

Feature creep. That is the spirit of this project. Because of personal projects I need the BBB need to be Zigbee enabled and preferably also wifi capable. Since I had already used the ESP32-C6 for such purposes I decided to use this.

\section{Features}

\begin{itemize}
    \item{Nine user programmable buttons}
    \item{ESP32-C6 - Zigbee, Wifi, BLE etc}
    \item{LiPo charger with Power Path}
    \item{USB-C sink IC}
    \item{LED lights}
\end{itemize}

\newpage
\tableofcontents

\newpage

\section{Hardware design}

The hardware architecture can be seen in figure \ref{fig:system-overview}. The design can definitely be made more simple and cheap, but that is not the purpose of this project. However, several of the parts can be skipped if not desired.

\subsection{Desired features}

\begin{itemize}
    \item{Zigbee}
    \item{Wifi}
    \item{Battery powered long life}
    \item{USB-C powered / charged}
    \item{LED lights}
\end{itemize}

\subsection{Optional Subsystems}

The following parts can be not-mounted or replaced with something else:
\begin{itemize}
    \item{USB-C Sink controller: Can be replaced by resistors.}
    \item{Fuel Gauge: Can be removed and the battery power can be bridged directly to the charger}
    \item{LiPo Charger + LiPo battery: Can be removed. System power directly bridged from VBUS.}
    \item{Boost regulator + LED Matrix: Can be skipped}
    \item{Miniature speaker: Can be skipped}
    \item{Flash: Can be avoided if using ESP32-C6FH4 or ESP32-C6FH8 instead}
\end{itemize}

\end{multicols}

\begin{figure}[H]
    \centering
    \includegraphics[width=1\linewidth]{system-diagram}
    \caption{The hardware architecture of the BBB}
    \label{fig:system-overview}
\end{figure}

\newpage

\begin{multicols}{2}

\subsection{Power requirements}

Before any parts of the circuit can be properly designed, we need to determine the power and current requirements throughout the system.

The maximum peak current consumption of the ESP32-C6 is during WiFi transmission, which can be up to \qty{354}{\milli\ampere} at \qty{3.3}{\volt} which is about \qty{1.2}{\watt} \cite[66]{esp32-c6-datasheet}. The average power consumption will be much lower than this, but we need to design the power system to be able to handle this peak.

Each WS2812 2020 LED consumes about $3 \times \qty{12}{\milli\ampere} \approx \qty{40}{\milli\ampere}$ at full brightness. Which for the full matrix of 9 LEDs would be about \qty{360}{\milli\ampere}, which at \qty{5}{\volt} would be \qty{1.8}{\watt} \cite{ws2812-product-page}.

A speaker such as \url{https://www.electrokit.com/en/miniatyrhogtalare-8ohm-1w-11x15mm} driven with a MAX98357A class D amplifier can consume up to \qty{1}{\watt} at \qty{5}{\volt}, which is about \qty{200}{\milli\ampere}. It is possible to run it directly from $V_{SYS}$ from the battery but it cannot output the full power. And the \qty{5}{\volt} boost regulator should be able to handle \qty{1.5}{\ampere} when fed with \qty{3.6}{\volt} from the battery.

\end{multicols}

\begin{table}[H]
    \centering
    \begin{tabular}{|l|r|r|r|}
        \hline
        \textbf{Subsystem}     & \textbf{Voltage} & \textbf{Peak Current}    & \textbf{Peak Power}       \\
        \hline
        ESP32-C6 (WiFi TX)     & \qty{3.3}{\volt} & \qty{354}{\milli\ampere} & \qty{1.2}{\watt}          \\
        LED Matrix (9x WS2812) & \qty{5}{\volt}   & \qty{360}{\milli\ampere} & \qty{1.8}{\watt}          \\
        Speaker (MAX98357A)    & \qty{5}{\volt}   & \qty{200}{\milli\ampere} & \qty{1.0}{\watt}          \\
        \hline
        \textbf{Total}         &                  &                          & \textbf{\qty{4.0}{\watt}} \\
        \hline
    \end{tabular}
    \caption{Peak current and power consumption summary}
    \label{tab:power-summary}
\end{table}

\begin{multicols}{2}

A \qty{500}{\mAh} battery might fit into the enclosure, and with an allowed discharge current of \qty{2}{C} it will only be able to provide max \qty{3.6}{\watt}. So not all things will be able to run at full power at the same time, but that is not a requirement for this project.


\subsection{Charger Circuit -- BQ25601}
Other candidates were MP2667, which has better quiescent current but no OTG and BQ25895 but it has D+/D- pins which I require for other purposes.

\subsection{USB-C Sink Controller -- FUSB303B}

\subsection{Fuel Gauge -- BQ27441-G1}
Another option might have been BQ27427, but it is bga.


\subsection{3V3 Buck-Boost Regulator -- TPS63900}

\subsection{5V Boost Regulator -- TPS61092}

\subsection{LED -- WS2812}

\subsection{Speaker amplifier -- MAX98357A}

\subsection{MCU -- ESP32-C6}

\subsubsection{Flash -- W25Q64JV}
The QSPI pin mapping to the ESP32 is described in the datasheet \cite[32]{esp32-c6-datasheet}.

Datasheet \cite{w25q64jv-datasheet}


\end{multicols}

\printbibliography

\end{document}
