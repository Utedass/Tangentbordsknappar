\documentclass[11px]{specification}

% Input encoding and typographical rules for English language
\usepackage[utf8]{inputenc}
\usepackage[english]{babel}
\usepackage[english]{isodate}

\usepackage{float}

\usepackage{lipsum}

\usepackage{fontspec}
\usepackage{newunicodechar}
\newfontfamily\symbola{Symbola}
\newcommand{\downzigzagarrow}{\begingroup\symbola ↯\endgroup}
\newunicodechar{↯}{\downzigzagarrow}


% These define global texts that are used in headers and titles.
\title{BBB, Bestest Button Brick \\- Design Description}
\author{Jonatan Gezelius}
\date{Februari 2026}
\revision{Revision 0.1}
\companylogo{↯ Jegatron Electronics ↯}



\begin{document}

\maketitle

\section{Overview}


\begin{multicols}{2}

\begin{colblock}
\includegraphics[width=0.8\linewidth]{BBB-photo}
\captionof{figure}{First iteration of BBB}
\label{fig:systemDiagram}
\end{colblock}

The BBB is a multipurpose user programmable input device and also serves as an exercise product to hone my skills and try out different circuit solutions. Not all features and component make sense from a real product perspective, but serves as a learning experience.

It all started with the desire to have a more professional feel to my hobby and learning projects. So instead of using the classical tactile buttons forced down into a breadboard entangled in jumper wires, I bought some random mechanical keyboard switches and 3d-printed an enclosure for them. Then came the time to make the circuit board to empower this.

Feature creep. That is the spirit of this project. Because of personal projects I need the BBB need to be Zigbee enabled and preferably also wifi capable. Since I had already used the ESP32-C6 for such purposes I decided to use this.

\section{Features}

\begin{itemize}
\item{Nine user programmable buttons}
\item{ESP32-C6 - Zigbee, Wifi, BLE etc}
\item{LiPo charger with Power Path}
\item{USB-C sink IC}
\item{LED lights}
\end{itemize}




\newpage
\tableofcontents

\newpage



\section{Hardware design}

The hardware architecture can be seen in figure \ref{fig:system-overview}. The design can definitely be made more simple and cheap, but that is not the purpose of this project. However, several of the parts can be skipped if not desired.

\subsection{Desired features}

\begin{itemize}
\item{Zigbee}
\item{Wifi}
\item{Battery powered long life}
\item{USB-C powered / charged}
\item{LED lights}
\end{itemize}

\subsection{Optional Subsystems}

The following parts can be not-mounted or replaced with something else:
\begin{itemize}
\item{USB-C Sink controller: Can be replaced by resistors.}
\item{Fuel Gauge: Can be removed and the battery power can be bridged directly to the charger}
\item{LiPo Charger + LiPo battery: Can be removed. System power directly bridged from VBUS.}
\item{Boost regulator + LED Matrix: Can be skipped}
\item{Miniature speaker: Can be skipped}
\item{Flash: Can be avoided if using ESP32-C6FH4 or ESP32-C6FH8 instead}
\end{itemize}

\end{multicols}

\begin{figure}[H]
\centering
\includegraphics[width=1\linewidth]{system-diagram}
\caption{The hardware architecture of the BBB}
\label{fig:system-overview}
\end{figure}

\newpage

\begin{multicols}{2}


\subsection{Charger Circuit -- MP2667}

\subsection{USB-C Sink Controller -- FUSB303B}

\subsection{Fuel Gauge -- BQ27427}

\subsection{3V3 Buck-Boost Regulator -- TPS63900}

\subsection{5V Boost Regulator -- TPS61092}

\subsection{LED -- WS2812}

\subsection{MCU -- ESP32-C6}





\end{multicols}


\end{document}
